\documentclass{article}

\usepackage[english]{babel}
\usepackage[utf8]{inputenc}
\usepackage{fancyhdr}

\usepackage[a4paper, total={7in, 10in}]{geometry}
\usepackage{graphicx}

\graphicspath{ {./images/} }

\pagestyle{fancy}
\fancyhf{}
\fancyhead[L]{\leftmark}
\fancyhead[R]{\thepage}

% New commands
    \newcommand{\vect}[1]{\left[\begin{array}{c}#1\end{array}\right]}
    \newcommand{\ifcases}[1]{\left\{\begin{array}{cc}\end{array}\right.}
% 

\title{Stepper Motors: Acceleration and torque curves}
\author{Samuel Nösslböck}
\date{October 2022}

\begin{document}

\maketitle

\section{Introduction}

\section{Motor torque and "Coil pendulum"}

Taking a stepper motor with two different coils A and B with the charging curve being defined as
\begin{equation}
    i_c(t) = I_{m} (1 - e^{\frac{t}{\tau}}) \quad \tau = \frac{I_{m}L}{U}
\end{equation}
with $I_M$ being the maximum current flowing though the coil, $L$ being the inductivity and $U$ being the volage applied to the coil.
Considering the torque has a linear proportionality to the current flowing though the coil:
\[
    T(t) \leftrightarrow i(t)
\]
the following 

\end{document}